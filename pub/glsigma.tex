\documentclass{article}
\usepackage{amsmath}
\newcommand{\ep}{\epsilon}
\newcommand{\Var}{\mathrm{Var}}
\title{$GL\sigma$-Models}
\author{Patrick O'Neill}
\begin{document}
\maketitle{}

\section{Introduction}
Elucidating the general principles by which transcriptional factors
(TFs) operate on short timescales, and evolve on long ones, is a
central task in computational biology.  One early effort towards this
goal was made by Schneider \textit{et al.}, who proposed that the
information content of a transcription factor binding motif ($R_{seq})
$ should equal the information content of the binding process itself,
measured as the difference between the entropies of the prior and
posterior distributions of the TF over the chromosome, given a binding
event ($R_{freq}$)\cite{schneider86}.  This claim was later taken up
again by Kim, Martinetz and Polani who argued for the approximate
equality of $R_{seq}$ and $R_{freq}$ with numerical evidence from an
essentially combinatorial model \cite{kim00}.

Empirically, however, the equality of $R_{seq}$ and $R_{freq}$ is far
from incontrovertible.  In fact, there is much stronger support for
the inequality $R_{seq} \leq R_{freq}$, with differences of as much as
10 bits observed in \textit{Escherichia coli}.  Why do such
discrepancies obtain?  The conceptual model provided in
\cite{schneider86} assumes a single copy of a TF which binds with
equal probability to one of $\gamma$ sites.  It is known, however,
that TFs bind with varying affinities to different sites, and that
these differences are even exploited by evolution to construct complex
genetic circuits, \textit{e.g.} regulons whose gene expression is
ordered temporally by binding site strength [XXX cite].  If sites are
bound differentially, though, the assumptions of the model in
\cite{schneider86} do not hold.

The aim of the present paper is to explore the effects of differential
regulation on relationship of $R_{seq}$ and $R_{freq}$.  To do so, we
present a simple probabilistic model of the behavior of a TF on a
chromosome which depends only on three parameters.  We offer this
generative model as a relatively tractable null hypothesis against
which one may test various claims about the kinetics and evolution of
TFs.

\section{Model Overview}

For simplicity, we assume that a TF binds to a site $s = s_1s_2\ldots
s_L$ of length $L$ with a free energy of binding $\epsilon(s) $ which
is linear and additive in each base, \textit{i.e.} 

\begin{equation}
  \label{eq:energy_scoring}
\epsilon(s) =
\sum_{i=1}^L\epsilon_i(s_i) =
\sum_{i=1}^L\sum_{b\in\{A,C,G,T\}}\epsilon_{ib}[s_i=b].
\end{equation}
We then consider the case of a single copy of the TF which binds to a
genome $g = g_1g_2\ldots g_G$of length $G$.  We define a
\textit{configuration} to be a pair $(\epsilon,g)$, \textit{i.e.} a
specific energy matrix and a specific genome.

We then define the ($G,L,\sigma$) distribution over all configurations
with genomes of length $G$ and energy matrices of length $L$ as the
distribution such that $\epsilon_{ib}\overset{i.i.d.}{\sim}
\mathcal{N}(0,\sigma)$ and $g_j\overset{i.i.d.}{\sim}
\mathcal{U}(\{A,C,G,T\})$, the discrete uniform distribution over the
set of nucleotides.  From this definition it follows trivially that
the probability of configuration $(\epsilon,G)$ under $(G,L,\sigma)$ is:

  \begin{equation}
    \label{eq:config_prob}
      P(\epsilon,G) = \frac{1}{4^G}\prod_{i=1}^L\prod_{b\in\{A,C,G,T\}}\phi(\epsilon_{ib};0,\sigma^2),
  \end{equation}
  where $phi(x;\mu,\sigma^2)$ is the normal p.d.f.  Note that the
  support of each $(G,L,\sigma)$ distribution is the set of all
  configurations of appropriate dimensions $G$ and $L$.

\subsection{Configuration statistics}
In this section we define several functions of a configuration which
will be of interest to us.  

First let us define the posterior entropy of a configuration.
Supposing that the off-rate is negligible, the stationary occupancy
probability for the $i$th site is given by a Boltzmann distribution:

\begin{equation}
  \label{eq:boltzmann}
  P(s_i\ \mathrm{is\ bound}) = \frac{e^{-\beta\epsilon(s_i)}}{Z}.
\end{equation}

The posterior Shannon entropy of the configuration is then:

\begin{equation}
\label{eq:post_ent}
H(\epsilon,G) = -\sum_jp_j\log_2p_j
\end{equation}

and the information content $R_{freq}$ of the binding event is given by:

\begin{equation}
  \label{eq:rfreq}
  R_{freq} = H_{prior} - H_{posterior} = \log_2(G) + \sum_jp_j\log_2p_j.
\end{equation}

\section{Model Analysis}
In this section we derive approximations for the posterior entropy of
a typical configuration in a $(G,L,\sigma)$ distribution.

\subsection{Site energy statistics}
\subsubsection{Mean site energy}
First, let us compute the expected energy of free binding for a given
energy matrix $\epsilon$.  If $\epsilon$ is fixed, then by taking the
expectation over sites, we have:
\begin{align*}
  <\epsilon(s)>_s =& \sum_{i=1}^L<\epsilon_i(s_i)>_{s_i}\\
  =& \frac{1}{4}\sum_{i=1}^L\epsilon_i\\.
\end{align*}

Then taking expectations over energy matrices, we obtain:
\begin{align*}
  <<\epsilon(s)>_s>_\epsilon =& <\frac{1}{4}\sum_{i=1}^L\epsilon_i>_\epsilon\\
  =& 0.
\end{align*}

\subsubsection{Variance of site energy}
The variance in the energy score of a given matrix with respect to
site is given by:
\begin{align*}
  \Var(\ep(s))_s =& <\ep(s)^2>_s - <\ep(s)>_s^2\\
  =& \sum_{i=1}^L(\frac{1}{4}\sum_{b\in\{A,C,G,T\}}\ep_{ib}^2 - \frac{1}{16}(\sum_{b\in\{A,C,G,T\}}\ep_{ib})^2)\\
  =& \sum_{i=1}^L(\frac{3}{16}\sum_{b\in\{A,C,G,T\}}\ep_{ib}^2 - \frac{1}{16}\sum_{b\neq b'}\ep_{ib}\ep_{ib'})\\
\end{align*}
from which it follows that:
\begin{align*}
  \Var(\ep(s))_{s,\ep} = L(\frac{3}{16}(4\sigma^2) - 0) = \frac{3}{4}L\sigma^2.
\end{align*}
We note that this expression is the expected variance in scores for a
large collection of random sites evaluated with an arbitrary
\textit{fixed} energy matrix, not a collection evaluated with a matrix
that randomly varies with the sites.  In the latter case, the variance
would be $L\sigma^2$: fixing the matrix with respect to the collection
reduces the variance by a factor of $\frac{1}{4}$.

\subsection{Posterior Entropy}
To estimate the posterior entropy of a `typical' configuration from the $(G,L,\sigma)$, we begin with the following identity from statistical mechanics:

\begin{align*}
  H =& -\sum_j p_j\log p_j\\
  =& -\sum_j \frac{e^{-\beta\ep_j}}{Z}\log \frac{e^{-\beta\ep_j}}{Z}\\
  =& \sum_j \frac{e^{-\beta\ep_j}}{Z}(\beta\ep_j + \log Z)\\
  =& \beta<\ep> + \log Z,\\
\end{align*}
where we take care to distinguish that the term $<\epsilon>$ is the
mean energy of the system, and not the expected score of the matrix.
If we can estimate these two terms, then clearly we will have an
estimate of the posterior entropy itself.

\subsubsection{Estimating the partition function}
In order to obtain those quantities, we must first estimate the
partition function $Z = \sum_je^{-\beta\ep_j}$.  Let us define
$X\sim-\beta\ep$, so that $<X>=0$ and
$\Var(X)=\frac{3}{4}\beta^2\sigma^2L$.  

Energies are normal: We give three arguments that
$X$ can be considered normally distributed, or more properly that our
uncertainty about $X$ is well-captured by a normal distribution.
Firstly, \textit{a priori} we know that $X$ is distributed as a sum of
$L$ independent random variables.  If $L$ is large enough, then $X$
will be approximately normally distributed.  Secondly, the mean and
variance of $X$ are the only known constraints on its distribution.
The principle of maximum entropy urges us to assume the distribution
which maximizes entropy subject to known constraints, which in the
case of known mean and variance is the normal distribution.  Thirdly,
computational studies suggest that the normal distribution is a
tolerable characterization of $X$ when $\mathcal{O}(L) = 10$.

If $X$ is normal, then $e^X$ is lognormal.  Hence, 
\begin{align*}
  <Z> =& G<\mathcal{LN}(0,\frac{3}{4}\beta^2\sigma^2L)>\\
  =&Ge^{\frac{3}{8}\beta^2\sigma^2L}
\end{align*}


\subsubsection{Mean energy ($<\ep>$)}


%  Assuming that
% $\epsilon_{ib}\overset{i.i.d.}{\sim} \mathcal{N}(0,\sigma)$,
% $g_j\overset{i.i.d.}{\sim} \mathcal{U}(\{A,C,G,T\})$ and $L$ is not
% too small.

\bibliography{refs/bibliography}{} \bibliographystyle{abbrv} \newpage
\end{document}
