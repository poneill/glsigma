\documentclass{article}
\usepackage{amsmath}
\newcommand{\ep}{\epsilon}
\newcommand{\logt}{\log_2}
\newcommand{\Var}{\mathrm{Var}}
\title{$GL\sigma$-Models}
\author{Patrick O'Neill}
\begin{document}
\maketitle{}

\section{Introduction}
We describe a mathematical model of the information content of a
transcription factor binding motif as a function of the regulatory
requirements of the TF-BS system.
\section{Assumptions}
We assume a genome containing $G$ potential binding sites in the 5'-3'
orientation only.  We assume that the genomic background is
well-approximated by a mononucleotide model.  For convenience we
assume the background base statistics are uniform, though this
assumption is easily relaxed.  The TF is present in a single copy and
binds to $w$-mers according to a simple match-mismatch energy model with
one preferred base at each position contributing $-\epsilon$ to the
binding energy.  The TF regulates $n$ sites of width $w$, and must
bind to the $i$th site at an occupancy $\alpha_i$.  Finally, we assume
the binding process is at equilibrium and suitable to description by
Boltzmann statistics at inverse temperature $\beta$.

\section{Model Derivation}
\subsection{Derivation of Binding Site Energies}
We first find the binding site energy $\epsilon_i$ required to bind
the site $s_i$ at an occupancy $\alpha_i$.  The background
contribution $Z_b$ to the partition function $Z$ can be
approximated\footnote{Approximating $Z_b$ by its expectation is always
  valid, because the first moment always exists.  The utility of the
  approximation depends on the distribution of $Z_b$ about its
  expectation, which is captured by the second moment.  We defer these
  questions to a later study.} as:

\begin{align}
  Z_b = \sum_{k=1}^Ge^{-\beta\epsilon_k}\approx& G<e^{-\beta\epsilon_k}> = G\omega^w  \label{eq:Zb}\\
  \omega =& \frac{3 + e^{\beta\epsilon}}{4}  \label{eq:omega}
\end{align}
where $\omega$ is the expected free energy for a background base.  The
components of the partition function $Z_f$ and $Z_b$ must satisfy the
relation:

\begin{equation}
  \label{eq:Zb_Zf}
  \frac{Z_f}{Z_f+Z_b} = \alpha,
\end{equation}
where $\alpha = \sum_i\alpha_i$, leading to the expression:
\begin{equation}
  \label{eq:Z}
  Z = \frac{Z_b}{1-\alpha}.
\end{equation}

From the assumption of Boltzmann statistics:

\begin{equation}
  \label{eq:boltzmann}
  \alpha_i = \frac{e^{-\beta\epsilon_i}}{Z}
\end{equation}
it then follows that:
\begin{equation}
  \label{eq:boltzmann}
   \epsilon_i= \frac{\log(\alpha_i) + \log(Z)}{-\beta}.
\end{equation}
The site energies $\epsilon_i$ are therefore functions only of the
site-occupancy $\alpha_i$, the total occupancy $\alpha$, and system
parameters $G,w,\epsilon,\beta$.
\subsection{Derivation of the match probability}
Given the site energies $\epsilon_i$, what is the probability of
observing a preferred base (a \textit{match}) in a column of the
binding motif?  Rescaling the site energies by the energy per match
gives us the expected number of matched bases $m_i$ in $s_i$, formally:

\begin{equation}
  \label{eq:expected_matches}
  m_i = \frac{\epsilon_i}{-\epsilon}.
\end{equation}
Rescaling again by the number of positions $w$ in the site, we obtain
the probability $p_i$ of observing a match in $s_i$:

\begin{equation}
  \label{eq:match_prob}
  p_i=  \frac{m_i}{w}.
\end{equation}
The probability of observing a match at a position chosen at random in
an arbitrary column of binding site alignment is therefore:
\begin{equation}
  \label{eq:p}
  p = \frac{1}{n}\sum_{i=1}^np_i.
\end{equation}

In effect we consider the motif as a joint distribution over a
collection of categorical random variables, assuming that the
positions of the binding motif are independent and identically
distributed.  In truth, the positions of the binding column are
naturally dependent through the relation $\epsilon_i
\sum_{j=1}^w\epsilon_{ij}$.  If $w$ is large in comparison to
$\epsilon$, though, random variates will tend to approximately
satisfy this relation and the dependence may be neglected.
\subsection{Derivation of the motif information content}
We consider each column of the binding motif as a random variable
emitting the preferred base with probability $p$, and each of the
three dispreferred bases with probability $\frac{1-p}{3}$.  The
entropy of the column is therefore:

\begin{equation}
  \label{eq:Hcol}
  H_{col} = -\left(p\logt p + (1-p)\logt\frac{1-p}{3}\right)
\end{equation}
On the assumption of independence between columns, the motif
information content $R_{seq}$ is given by:
\begin{equation}
  \label{eq:Rseq}
  R_{seq} = w(H_{prior} - H_{col})
\end{equation}
pwhere $H_{prior}$ is 2 bits in the case of a uniform background.

This completes the derivation of the model.  We next explore the
relationship between $R_{seq}$ and the model inputs.

\section{Model Analysis}
In this section we explore the dependence of $R_{seq}$ on the model
parameters.  First, we see that:
\begin{equation}
  \label{eq:plimit}
  \frac{2 \log\omega}{\beta \epsilon} + \frac{2 \log{\left (G \right )}}{\beta \epsilon w} + \frac{\log{\left (\alpha_{i} \right )}}{2 \beta \epsilon w} + \frac{3 \log{\left (\alpha_{o} \right )}}{2 \beta \epsilon w} + \frac{1}{\beta \epsilon w} \log{\left (\frac{1}{- \alpha + 1} \right )} + \frac{1}{\beta \epsilon w} \log{\left (- \alpha_{i} - \alpha_{o} + 1 \right )} + \frac{\log{\left (\alpha_{i} \right )}}{\beta \epsilon n w} - \frac{\log{\left (\alpha_{o} \right )}}{\beta \epsilon n w}
\end{equation}
\end{document}
